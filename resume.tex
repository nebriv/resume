%!TEX program = lualatex
%% The MIT License (MIT)
%%
%% Copyright (c) 2015 Daniil Belyakov
%%
%% Permission is hereby granted, free of charge, to any person obtaining a copy
%% of this software and associated documentation files (the "Software"), to deal
%% in the Software without restriction, including without limitation the rights
%% to use, copy, modify, merge, publish, distribute, sublicense, and/or sell
%% copies of the Software, and to permit persons to whom the Software is
%% furnished to do so, subject to the following conditions:
%%
%% The above copyright notice and this permission notice shall be included in all
%% copies or substantial portions of the Software.
%%
%% THE SOFTWARE IS PROVIDED "AS IS", WITHOUT WARRANTY OF ANY KIND, EXPRESS OR
%% IMPLIED, INCLUDING BUT NOT LIMITED TO THE WARRANTIES OF MERCHANTABILITY,
%% FITNESS FOR A PARTICULAR PURPOSE AND NONINFRINGEMENT. IN NO EVENT SHALL THE
%% AUTHORS OR COPYRIGHT HOLDERS BE LIABLE FOR ANY CLAIM, DAMAGES OR OTHER
%% LIABILITY, WHETHER IN AN ACTION OF CONTRACT, TORT OR OTHERWISE, ARISING FROM,
%% OUT OF OR IN CONNECTION WITH THE SOFTWARE OR THE USE OR OTHER DEALINGS IN THE
%% SOFTWARE.
% The font could be set to Windows-specific Calibri by using the 'calibri' option
\documentclass[]{mcdowellcv}

% For mathematical symbols
\usepackage{amsmath}


% Set applicant's personal data for header
\name{Ben Virgilio}
\address{New York, New York}
\contacts{ben@benvirgilio.com \linebreak https://benvirgilio.com}


\begin{document}
% Print the header
\makeheader

\begin{cvsection}{Experience}

    \begin{jobentry}{American Museum of Natural History}{New York, New York}
    \begin{jobpositions}
        \jobposition{Cybersecurity Manager}{May 2022 - Present}
        \jobposition{Senior Security Engineer}{August 2017 - May 2022}
        \jobposition{Security Engineer}{February 2016 - August 2017}
    \end{jobpositions}
        \jobdescription{Led the development of the cybersecurity program at a globally recognized research and educational institution. Implemented layered defenses emphasizing increased visibility and rapid detection, utilizing both emerging and proven technologies. Prioritized strategic, impactful security measures that both safeguarded and supported the institution’s primary focus on research and education. These security enhancements enabled the institution to pursue its global mission with confidence and assurance.}
        
        \begin{jobresponsibilities}
            \item Formulate and oversee the institution's cybersecurity strategy, ensuring a balance between security measures and the organization's core focus on research and education.
            \item Continuously assess potential security risks and implement measures to mitigate vulnerabilities across on-premises infrastructure, cloud platforms, and vendor solutions.
            \item Architect and deploy a multi-tiered security approach, incorporating both traditional defenses like network-based IDS and advanced deceptive measures. This strategy aims to enhance detection speed, increase system-wide visibility, and proactively identify potential security incidents in their earliest stages.
            \item Lead the institution's response to cybersecurity incidents, coordinating rapid detection, containment, and remediation efforts.
            \item Influence the selection, deployment, and effective use of security technologies and tools in alignment with the institution's objectives.
            \item Assist the CISO in the development, review, and refinement of cybersecurity policies and procedures. Ensure alignment with relevant regulations and security frameworks, and facilitate security audits, liaising with auditors as required.
            \item Engage with senior management, departments, and external industry entities, facilitating effective communication and collaboration on security matters. Participate in the Information Sharing and Analysis Center (ISAC) specific to educational institutions.
        \end{jobresponsibilities}
        
        \begin{jobprojects}
            \item Supported the deployment of an NSF Grant Funded (Award \#1827153) Science DMZ network, a high-speed network designed to support the transfer and management of high-volume scientific research data. Responsible for the development and implementation of low impact, highly effective security measures.
            \item Spearheaded the integration of 802.1x EAP-TLS authentication, complemented by the implementation of Single Sign-On (SSO), advanced Identity and Access Management (IAM), and Azure Conditional Access. Further fortified user access controls through the deployment of DuoSecurity's Multi-Factor Authentication (MFA) and FIDO2 tokens, driving the institution's move toward a holistic, risk based, zero-trust framework.
            \item Authored over 60 internal code repositories in GitLab, comprised of tools for operations, incident investigation, IOC automation, SIEM mailbox monitoring, data redaction, and more. Many repositories employ CI/CD processes to ensure successful deployment of critical detection and prevention solutions.
            \item Transitioned twelve legacy ASA firewalls to a robust multi-context high-availability cluster. Additionally, introduced cloud-based firewalls, facilitating the secure migration of essential business processes to cloud infrastructures, offering scalability and adaptability to emerging threats.
            \item Developed an in-house security operations platform using open-source and cloud solutions, bolstering threat monitoring and incident handling. Implemented Defender Advanced Hunting rules tailored for sophisticated phishing campaigns, and other industry specific threats. Integrated detection and response strategies with the Mitre ATT\&CK framework, ensuring comprehensive coverage and minimizing gaps in institutional defenses against adversary tactics.
            \item Led the evolution of the institution's centralized logging capabilities by implementing an ELK stack. Directed the integration of diverse log sources, including servers, network hardware, cloud services, and more. Developed unit testing for specific log sources to validate and ensure consistency across system upgrades and changes. Strategically developed alerts based on data collected from previous penetration tests, bolstering detection and proactive response to potential attack vectors.
            \item Developed a tailored cybersecurity training curriculum, complemented by controlled phishing tests to gauge and enhance employee threat awareness. This initiative led to heightened user vigilance, evidenced by the successful detection and reporting of sophisticated phishing campaigns.
            \item Transitioned from Spirion to Microsoft's DLP solution to enhance secure management of sensitive data using on-premise data scanners. Collaborated with appointed departmental data custodians to review and address findings, streamlining data protection and compliance efforts. Developed sensitive data labels for users to flag their data in accordance with institutional data security policies. 
        \end{jobprojects}
    \end{jobentry}

    \begin{jobentry}{Stroz Friedberg}{New York, New York}
        \begin{jobpositions}
            \jobposition{Associate}{January 2015 - February 2016}
            \jobposition{Analyst}{June 2014 - January 2015}
        \end{jobpositions}
        
        \begin{jobresponsibilities}
            \item Conducted comprehensive security assessments, emphasizing a holistic approach that not only identified technical vulnerabilities but also systemic issues rooted in organizational politics and culture.
            \item Executed penetration tests on wireless, physical infrastructure, servers, and web apps with tools such as Metasploit, BurpSuite, and Responder. Highlighted how vulnerabilities were chained together to escalate from unprivileged access to domain admin roles.
            \item Presented technical findings to C-suite, translating complex vulnerabilities into actionable insights. Collaborated with client technical teams to prioritize and tailor solutions, ensuring recommendations fit their unique environment and needs.
            \item Supported major incident response engagements, leveraging offensive security experience to provide detailed roadmaps of potential attack vectors and vulnerabilities.
            \item Engaged in reverse engineering of zero-day vulnerabilities identified during assessments, subsequently reporting findings to relevant vendors for mitigation.
        \end{jobresponsibilities}
        
        \begin{jobprojects}
            \item Developed internal scripts to assess client Active Directory environments for vulnerabilities. These scripts facilitated rapid, accurate, and repeatable insights into the health of the client's systems, predicting broader infrastructural challenges. Automatically generated actionable reporting data, significantly enhancing the efficiency of final client report preparations.
            \item Supported the creation, deployment, and implementation of a portable log analytics system, streamlining forensic and incident response investigations.
        \end{jobprojects}
    \end{jobentry}

    \begin{jobentry}{Senator Patrick Leahy Center for Digital Investigation}{Burlington, Vermont}
        \begin{jobpositions}
            \jobposition{Lead Network Administrator}{September 2013 - June 2014}
        \end{jobpositions}
        
        \jobdescription{Managed and maintained the research and isolated forensics networks, ensuring data integrity for active case files. Additionally, provided insights and resources to LCDI research projects and oversaw the internal web resources for employees.}
    \end{jobentry}

    \begin{jobentry}{UMass Amherst (Administration and Finance)}{Amherst, Massachusetts}
        \begin{jobpositions}
            \jobposition{Security Consultant}{January 2012 - August 2013}
        \end{jobpositions}
        
        \jobdescription{Conducted NTFS permissions audits on departmental file shares, ensuring optimal access levels. Located sensitive data using Identity Finder, collaborated with data custodians on remediation. Created vulnerability remediation plans via the Qualys platform. Established and reviewed secure configurations for multi-function printers, developing automated tools for continuous assessment and remediation.}

    \end{jobentry}

\end{cvsection}

\begin{cvsection}{Education}
    \begin{cvsubsection}{Burlington, VT}{Champlain College}{2010 | 2014}
        \vspace{-1em}Bachelor of Science Degree in Computer Networking and Information Security\\
        \hspace*{1em}With Minor in Computer and Digital Forensics
    \end{cvsubsection}
\end{cvsection}

\begin{cvsection}{Project Highlights}
    \begin{cvsubsection}{}{}{}
        Most of my work can be found on \href{https://github.com/nebriv/}{my GitHub account https://github.com/nebriv/}, but there are a few projects I wanted to highlight.\\
        \begin{itemize}
            \setlength\itemsep{3pt}
            \item \textbf{This Resume!} (https://github.com/nebriv/resume/)
            \\Constructed this resume using LaTeX, integrating it with a custom-built GitHub action workflow for automated PDF generation. This implementation ensures efficient versioning and flexibility for content variations.
            \item \textbf{VTOLVR-Mods} (https://vtolvr-mods.com)
            \\Developed python backend to allowing mod creators to upload their mod packages to a central location accessible by the mod loader platform. Site features include a RESTful API, numerous integrations with Steam, Discord and others, as well as full administration and moderation utilizes.
            \item \textbf{Home Lab/Home Automation}
            \\Deployed internal Proxmox hypervisor to run home lab resources including standard network requirements (DNS/IDS/etc), along with Home Assistant and other home automation utilities. Multiple VLANs separate IOT/media/computer systems, bridged with an OPNSense router out to commodity ISP. Home lab provides a sandbox for deployment of various configurations and software stacks. This setup also provides VPN integrations to various cloud providers for added flexibility. Primary goal of this home lab is to serve as a dynamic platform for ongoing education, and as a manifestation of my intrinsic drive to understand, automate, and innovate.
        \end{itemize}
    \end{cvsubsection}
\end{cvsection}

\begin{cvsection}{Skills/Proficiencies}
    \begin{cvsubsection}{}{}{}
        \begin{itemize}
            \item \textbf{Cisco Products}  ASA, AnyConnect, Identity Services Engine (ISE), Umbrella, DuoSecurity
            \item \textbf{Scripting/Programming Languages}  Python, PHP, Powershell
            \item \textbf{Microsoft E5/A5 Solutions}  Defender for X, Entra, Sentinel, Purview
            \item \textbf{Microsoft Azure} Azure Firewall, Sentinel, Container Apps, Logic Apps, Log Analytics, Virtual Machines and Networking
            \item \textbf{Amazon Web Services (AWS)} IAM Identity Center, EC2, VPC, Lambda, Route 53, S3 
            \item \textbf{Operating Systems} Windows Server, RHEL/CentOS, Debian/Ubuntu
            \item \textbf{Vulnerability Scanners}  Acunetix, Tenable.io/Nessus, Qualys, BurpSuite, Metasploit, Responder, sqlMap, and many more open source security tools as they are released
        \end{itemize}
    \end{cvsubsection}
\end{cvsection}

\begin{cvsection}{Certifications}
	\begin{cvsubsection}{}{}{}
		\begin{itemize}
			\setlength\itemsep{3pt}
			\item \textbf{CompTIA Security+ Certified} \hfill July 2013 – July 2016
			\item \textbf{GIAC Continuous Monitoring} \hfill  September 2017 – Present
            \item \textbf{GIAC Defensible Security Architecture} \hfill  October 2021 – Present
            \item \textbf{GIAC Public Cloud Security October} \hfill  2022 – Present
		\end{itemize}
	\end{cvsubsection}
\end{cvsection}

\customfooter
\end{document}
